%% ============================================================================
%% AI Research Template - Theoretical Analysis Template
%% Author: Viska Wei
%% ============================================================================

\documentclass[11pt]{article}

%% Packages
\usepackage{amsmath,amssymb,amsfonts,amsthm}
\usepackage[margin=1in]{geometry}
\usepackage{hyperref}
\usepackage{enumitem}
\usepackage{bm}
\usepackage{tcolorbox}

%% Theorem environments
\newtheorem{theorem}{Theorem}[section]
\newtheorem{proposition}[theorem]{Proposition}
\newtheorem{lemma}[theorem]{Lemma}
\newtheorem{corollary}[theorem]{Corollary}
\newtheorem{definition}[theorem]{Definition}
\newtheorem{remark}[theorem]{Remark}
\newtheorem{example}[theorem]{Example}
\newtheorem{assumption}{Assumption}

%% Commands
\newcommand{\R}{\mathbb{R}}
\newcommand{\E}[1]{\mathbb{E}\left[{#1}\right]}
\newcommand{\Var}{\mathrm{Var}}
\newcommand{\Cov}{\mathrm{Cov}}
\newcommand{\argmin}{\operatorname{argmin}}
\newcommand{\argmax}{\operatorname{argmax}}

\title{\textbf{Theoretical Analysis for {{PROJECT_NAME}}}\\[0.5em]
\large {{SUBTITLE}}}
\author{Viska Wei}
\date{\today}

\begin{document}
\maketitle

\begin{abstract}
This document presents a comprehensive theoretical analysis for {{PROJECT_DESCRIPTION}}. 
We establish: 
(i) identifiability conditions, 
(ii) consistency and convergence rates for the estimator, 
(iii) minimax lower bounds proving optimality, and 
(iv) neural network approximation and generalization bounds (if applicable).
\end{abstract}

\tableofcontents
\newpage

%% ============================================================================
\section{Problem Formulation}
\label{sec:formulation}
%% ============================================================================

\subsection{Notation and Setup}

Let $\mathcal{X}$ be the data space and $\mathcal{H}$ be the hypothesis space. 

\begin{itemize}
    \item[(A1)] \textbf{Data Model}: {{DATA_MODEL_ASSUMPTION}}
    \item[(A2)] \textbf{Regularity}: {{REGULARITY_ASSUMPTION}}
    \item[(A3)] \textbf{Identifiability}: {{IDENTIFIABILITY_ASSUMPTION}}
\end{itemize}

\subsection{Loss Function}

Define the population loss:
\begin{equation}\label{eq:pop_loss}
\mathcal{L}(\theta) := \E{\ell(\theta; X)},
\end{equation}
where $\ell$ is the per-sample loss function.

The empirical loss with $n$ samples:
\begin{equation}\label{eq:emp_loss}
\hat{\mathcal{L}}_n(\theta) := \frac{1}{n}\sum_{i=1}^{n} \ell(\theta; X_i).
\end{equation}

%% ============================================================================
\section{Identifiability Analysis}
\label{sec:identifiability}
%% ============================================================================

\begin{definition}[Identifiability]\label{def:identifiability}
A parameter $\theta^*$ is \emph{identifiable} if for any $\theta \in \Theta$:
$$
\mathcal{L}(\theta) = \mathcal{L}(\theta^*) \implies \theta = \theta^* \text{ (up to equivalence)}.
$$
\end{definition}

\begin{theorem}[Identifiability Condition]\label{thm:identifiability}
Under assumptions (A1)-(A3), if the following coercivity condition holds:
\begin{equation}\label{eq:coercivity}
\mathcal{L}(\theta) - \mathcal{L}(\theta^*) \geq c_H \|\theta - \theta^*\|^2,
\end{equation}
for some $c_H > 0$, then $\theta^*$ is identifiable.
\end{theorem}

\begin{proof}
% Proof here
\end{proof}

%% ============================================================================
\section{Consistency and Convergence Rates}
\label{sec:consistency}
%% ============================================================================

\begin{theorem}[Consistency]\label{thm:consistency}
Let $\hat{\theta}_n = \argmin_{\theta \in \Theta} \hat{\mathcal{L}}_n(\theta)$. Under assumptions (A1)-(A3), as $n \to \infty$:
$$
\hat{\theta}_n \xrightarrow{P} \theta^*.
$$
\end{theorem}

\begin{theorem}[Convergence Rate]\label{thm:rate}
Under the additional assumption that $\Theta$ has finite complexity (e.g., finite VC dimension), we have:
\begin{equation}
\E{\|\hat{\theta}_n - \theta^*\|^2} \leq \frac{C}{c_H^2} \cdot \frac{d}{n},
\end{equation}
where $d$ is the effective dimension and $C$ is a universal constant.
\end{theorem}

%% ============================================================================
\section{Minimax Lower Bounds}
\label{sec:lower_bounds}
%% ============================================================================

\begin{theorem}[Information-Theoretic Lower Bound]\label{thm:lower_bound}
For the estimation problem, the minimax risk satisfies:
\begin{equation}
\inf_{\hat{\theta}} \sup_{\theta^* \in \Theta} \E{\|\hat{\theta} - \theta^*\|^2} \geq \frac{c \cdot d}{n},
\end{equation}
where $c > 0$ is a constant depending only on the problem structure.
\end{theorem}

\begin{proof}
The proof uses Fano's inequality or Le Cam's method. 
% Details here
\end{proof}

\begin{corollary}[Optimality]
The rate $n^{-1}$ achieved by our estimator is minimax optimal.
\end{corollary}

%% ============================================================================
\section{Neural Network Bounds (if applicable)}
\label{sec:nn_bounds}
%% ============================================================================

For neural network estimators, we decompose the error.

\begin{theorem}[Approximation Error]\label{thm:nn_approx}
Let $\mathcal{F}_{NN}(W, D)$ denote ReLU networks with width $W$ and depth $D$. If $\theta^* \in C^s$, then:
\begin{equation}
\inf_{f \in \mathcal{F}_{NN}} \|f - \theta^*\|_\infty \leq C W^{-2s/d} (\log W)^{2s/d}.
\end{equation}
\end{theorem}

\begin{theorem}[Generalization Bound]\label{thm:generalization}
The total error satisfies:
\begin{equation}
\E{\mathcal{L}(\hat{\theta}_{NN})} - \mathcal{L}(\theta^*) \leq \underbrace{O(W^{-2s/d})}_{\text{approximation}} + \underbrace{O\left(\frac{WD \log(WD)}{n}\right)}_{\text{estimation}}.
\end{equation}
\end{theorem}

%% ============================================================================
\section{Fisher Information and CRLB}
\label{sec:fisher}
%% ============================================================================

\begin{definition}[Fisher Information]\label{def:fisher}
The Fisher information matrix at $\theta$ is:
\begin{equation}
I(\theta) := \E{\nabla_\theta \log p(X; \theta) \nabla_\theta \log p(X; \theta)^\top}.
\end{equation}
\end{definition}

\begin{theorem}[Cramér-Rao Lower Bound]\label{thm:crlb}
For any unbiased estimator $\hat{\theta}$:
\begin{equation}
\Var(\hat{\theta}) \geq I(\theta)^{-1}.
\end{equation}
\end{theorem}

\begin{tcolorbox}[colback=blue!5!white,colframe=blue!75!black,title=Practical Implications]
The CRLB tells us:
\begin{itemize}
    \item \textbf{Theoretical ceiling}: No estimator can beat $I(\theta)^{-1}$ variance
    \item \textbf{Data requirement}: To achieve error $\epsilon$, need $n \geq \text{tr}(I^{-1}) / \epsilon^2$
    \item \textbf{Efficiency}: If our estimator achieves CRLB, it is statistically optimal
\end{itemize}
\end{tcolorbox}

%% ============================================================================
\section{Summary}
\label{sec:summary}
%% ============================================================================

\begin{table}[h]
\centering
\caption{Theoretical Results Summary}
\begin{tabular}{lcc}
\toprule
\textbf{Result} & \textbf{Rate} & \textbf{Optimal?} \\
\midrule
Upper Bound & $O(d/n)$ & -- \\
Lower Bound & $\Omega(d/n)$ & -- \\
\textbf{Gap} & \textbf{None} & \textbf{Yes} \\
\bottomrule
\end{tabular}
\end{table}

\bibliographystyle{plain}
\bibliography{theory}

%% ============================================================================
\appendix
\section{Proof Details}
\label{app:proofs}
%% ============================================================================

% Detailed proofs here

\section{Technical Lemmas}
\label{app:lemmas}

% Supporting lemmas

\end{document}
